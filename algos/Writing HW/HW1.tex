\documentclass[11pt,a4paper,oneside]{article}
\usepackage[a4paper, margin=1in, bottom=0.75in]{geometry}

\usepackage[T2A]{fontenc}
\usepackage[utf8]{inputenc}
\usepackage[english,russian]{babel}
\usepackage{xcolor}
\usepackage{ulem}
\usepackage{soulutf8}
\usepackage{soul}
\usepackage{fancyhdr}
\usepackage{amsmath}
\usepackage{amssymb}
\usepackage[shortlabels]{enumitem}
\usepackage{titlesec}
\usepackage{hyperref}
\usepackage{multicol}
\usepackage[most]{tcolorbox}
\usepackage{mdframed}
\usepackage{listings}
\usepackage[makeroom]{cancel}
\usepackage{tocloft}

\title{HW1}
\author{Dima Nartov}
\date{September 2022}

\begin{document}

\maketitle

\section{Task 1}
    Если $f_1(n) = \mathcal{O}(g_1(n))$ и $f_2(n) = \mathcal{O}(g_2(n))$, то $f_1(n) + f_2(n) = \mathcal{O}(g_1(n) + g_2(n))$.
    \\
    $f_1(n) \leq c_1 \cdot g_1(n)$
    \\
    $f_2(n) \leq c_2 \cdot g_2(n)$
    \\
    $f_1(n) + f_2(n) \leq c_3 \cdot (g_1(n) + g_2(n))$
    \\
    $c_1 \cdot g_1(n) + c_2 \cdot g_2(n) \leq c_3 \cdot (g_1(n) + g_2(n))$ При $c_3 = c_1 + c_2$ имеем верно равенство - утверждение доказано.
\section{Task 2}
    $\max(f(n), g(n)) = \Theta(f(n) + g(n))$
    \\
    $c_1 \cdot (f(n) + g(n)) \leq \max(f(n), g(n)) \leq c_2 \cdot (f(n) + g(n))$
    \\
    При $c_1 = \frac{1}{2}$ и $c_2 = 1$ утверждение верно, так максимум - дает максимальное из двух, следовательно второе число либо равно, либо меньше. Наибольшую сумму мы будем получать, если $f(n) = g(n)$, поэтому поделив на $2$, мы получим имеено максимум. Точно также оцениваем сверху.
\section{Task 3}
    $\sum \limits_{i=1}^{n + 5} 2^i = \frac{2(2^{(n+5)} - 1)}{2 - 1} = 2^{(n+6)}-2$
    \\
    $2^{(n+6)}-2 = \mathcal{O}(2^n)$
    \\
    $2^{(n+6)}-2 \leq c \cdot 2^n$ 
    При $2^7$ получаем верное равенство!
\section{Task 4}
    $\frac{n^3}{6} - 7n^2 = \Omega(n^3)$
    \\
    $\frac{n^3}{6} - 7n^2 \geq c \cdot n^3$
    \\
    $c \geq \frac{42 - n}{6n}$
    Всегда существует положительное $c$, значит утверждение доказано!
\section{Task 5}
    $1$, $(\frac{3}{2})^2$, $n^{\frac{1}{\log n}}$, $\log \log n$, $\sqrt{\log n}$, $\log^2 n$, $(\sqrt{2})^{\log n}$, $n$, $2^{\log n}$, $\log (n!)$, $n \log n$, $n^2$, $4^{\log n}$, $n^3$, $(\log n)!$, $n^{\log \log n}$, $(\log n)^{\log n}$, $n \cdot 2^n$, $e^n$, $n!$, $(n + 1)!$, $2^{2^n}$, $2^{2^{n + 1}}$
\end{document}


